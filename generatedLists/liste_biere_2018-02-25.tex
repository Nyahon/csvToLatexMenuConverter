%%%%%%%%%%%%%%%%%%%%%%%%%%%%%%%%%%%%%%%%%
% Medium Length Professional CV
% LaTeX Template
% Version 2.0 (8/5/13)
%
% This template has been downloaded from:
% http://www.LaTeXTemplates.com
%
% Original author:
% Trey Hunner (http://www.treyhunner.com/)
%
% Important note:
% This template requires the resume.cls file to be in the same directory as the
% .tex file. The resume.cls file provides the resume style used for structuring the
% document.
%
%%%%%%%%%%%%%%%%%%%%%%%%%%%%%%%%%%%%%%%%%
%
%----------------------------------------------------------------------------------------
%	PACKAGES AND OTHER DOCUMENT CONFIGURATIONS
%----------------------------------------------------------------------------------------
\documentclass{resume} % Use the custom resume.cls style
%
\usepackage[french]{babel}
\usepackage[utf8]{inputenc}
\usepackage[T1]{fontenc}
\usepackage{textcomp}
\usepackage{lmodern}
\usepackage{graphicx}
\usepackage[top=0.6in, bottom=0.6in, left=0.75in, right=0.75in]{geometry} % Document margins

\name{Liste des bières}

\renewcommand{\familydefault}{\sfdefault}

\begin{document}

	\begin{rSection}{Abbaye des Trappistes de Westmalle}

	{\bf Westmalle Triple} – {\em Blonde Trappiste } – {\em 9.5\% Vol.} \hfill {\em 3 CHF} \\

	\end{rSection}

	\begin{rSection}{Achouffe}

	{\bf La Chouffe} – {\em Blonde } – {\em 8\% Vol.} \hfill {\em 3 CHF} \\

	\end{rSection}

	\begin{rSection}{Bière du Sorcier}

	{\bf Bière du Sorcier} – {\em Spéciale Epice } – {\em 5\% Vol.} \hfill {\em 3 CHF} \\

	\end{rSection}

	\begin{rSection}{Blue Moon}

	{\bf Blue Moon blanche} – {\em Blanche } – {\em 5.4\% Vol.} \hfill {\em 3 CHF} \\

	\end{rSection}

	\begin{rSection}{Bosteels}

	{\bf Kwak} – {\em Belgian Pale Ale } – {\em 8.4\% Vol.} \hfill {\em 3 CHF} \\

	{\bf Karmeliet} – {\em Triple } – {\em 8.4\% Vol.} \hfill {\em 3 CHF} \\

	\end{rSection}

	\begin{rSection}{Boxer}

	{\bf Boxer Old} – {\em Blonde } – {\em 5.2\% Vol.} \hfill {\em 1 CHF} \\
	La classique boxer qui n'est plus à présenter ! \\

	\end{rSection}

	\begin{rSection}{Brasserie du Jorat}

	{\bf La Blanche} – {\em Blanche } – {\em 5.5\% Vol.} \hfill {\em 3 CHF} \\

	{\bf L'ambrée} – {\em Ambrée } – {\em 6\% Vol.} \hfill {\em 3 CHF} \\

	{\bf La Noire} – {\em Noire } – {\em 5.5\% Vol.} \hfill {\em 3 CHF} \\

	\end{rSection}

	\begin{rSection}{Brewdog}

	{\bf Dead Pony Club} – {\em Pale Ale } – {\em 3.8\% Vol.} \hfill {\em 3 CHF} \\

	{\bf Punk IPA} – {\em Post Modern Classic } – {\em 5.6\% Vol.} \hfill {\em 3 CHF} \\

	{\bf King Pin} – {\em Lager } – {\em 4.7\% Vol.} \hfill {\em 3 CHF} \\

	{\bf Jack Hammer} – {\em IPA } – {\em 7.2\% Vol.} \hfill {\em 3 CHF} \\

	{\bf Nanny State} – {\em Spéciale } – {\em 0.5\% Vol.} \hfill {\em 3 CHF} \\

	\end{rSection}

	\begin{rSection}{Docteur Gab's}

	{\bf Houleuse} – {\em Blanche } – {\em 5\% Vol.} \hfill {\em 4 CHF} \\

	{\bf Chameau} – {\em Ambrée } – {\em 7\% Vol.} \hfill {\em 4 CHF} \\

	{\bf IPAnema} – {\em IPA } – {\em 6\% Vol.} \hfill {\em 4 CHF} \\

	{\bf Pépite} – {\em Blonde Pale Ale } – {\em 4.8\% Vol.} \hfill {\em 4 CHF} \\

	{\bf Tempête} – {\em Blonde Triple } – {\em 8\% Vol.} \hfill {\em 4 CHF} \\

	{\bf Ténébreuse} – {\em Stout Noire } – {\em 6\% Vol.} \hfill {\em 4 CHF} \\

	{\bf Crafty} – {\em Cidre } – {\em 5\% Vol.} \hfill {\em 4 CHF} \\

	{\bf Swaf} – {\em Blonde } – {\em 4.8\% Vol.} \hfill {\em 2 CHF} \\

	\end{rSection}

	\begin{rSection}{Erdinger Weissbräu}

	{\bf Erdinger Weissbier} – {\em Blanche } – {\em 5.3\% Vol.} \hfill {\em 2 CHF} \\

	\end{rSection}

	\begin{rSection}{Hoegaarden}

	{\bf Hoegaarden} – {\em Blanche } – {\em 4.9\% Vol.} \hfill {\em 2 CHF} \\
	Blanche délicieuse et facile, un ultra-classique. Pour ceux qui veulent se détendre sans réfléchir. \\

	\end{rSection}

	\begin{rSection}{Landi}

	{\bf Farmer blanche} – {\em Blanche } – {\em 5.3\% Vol.} \hfill {\em 1 CHF} \\

	{\bf Farmer blonde} – {\em Blonde } – {\em 4.5\% Vol.} \hfill {\em 1 CHF} \\

	\end{rSection}

	\begin{rSection}{Le Rouget de L'isle}

	{\bf Millefleur} – {\em Blonde Cervoise } – {\em 6.4\% Vol.} \hfill {\em 3 CHF} \\

	{\bf Fourche du Diable} – {\em Cervoise } – {\em 6\% Vol.} \hfill {\em 3 CHF} \\

	{\bf Vieux Tuyé} – {\em Fumée } – {\em 6\% Vol.} \hfill {\em 3 CHF} \\

	{\bf Blanche des Plateaux} – {\em Blanche } – {\em 4.8\% Vol.} \hfill {\em 3 CHF} \\

	\end{rSection}

	\begin{rSection}{Leffe}

	{\bf Leffe Brune} – {\em Brune } – {\em 6.5\% Vol.} \hfill {\em 3 CHF} \\

	{\bf Leffe Blonde} – {\em Blonde } – {\em 6.6\% Vol.} \hfill {\em 3 CHF} \\

	{\bf Leffe Vieille Cuvée} – {\em Ambrée } – {\em 8.2\% Vol.} \hfill {\em 3 CHF} \\

	\end{rSection}

	\begin{rSection}{Lindemans}

	{\bf Lindemans Kriek} – {\em Lambic } – {\em 3.5\% Vol.} \hfill {\em 2 CHF} \\

	{\bf Faro} – {\em Lambic } – {\em 4.5\% Vol.} \hfill {\em 2 CHF} \\

	\end{rSection}

	\begin{rSection}{Tennent Caledonian}

	{\bf Tennent's} – {\em Whisky } – {\em 6\% Vol.} \hfill {\em 3 CHF} \\

	\end{rSection}

	\begin{rSection}{Unibroue}

	{\bf La Fin Du Monde} – {\em Triple } – {\em 9\% Vol.} \hfill {\em 3 CHF} \\

	{\bf Maudite} – {\em Double } – {\em 8\% Vol.} \hfill {\em 3 CHF} \\

	{\bf Blonde de Chambly} – {\em Blonde } – {\em 5\% Vol.} \hfill {\em 3 CHF} \\

	\end{rSection}

	\begin{rSection}{Wittekop}

	{\bf Wittekop} – {\em Blanche } – {\em 4.5\% Vol.} \hfill {\em 2 CHF} \\

	\end{rSection}

	\newpage
	\begin{rSection}{Softs}

	{\bf Nestea Peach} \hfill {\em 1 CHF} \\

	{\bf Nestea Lemon} \hfill {\em 1 CHF} \\

	{\bf Coca Cola} \hfill {\em 1 CHF} \\

	{\bf Red Bull} \hfill {\em 2 CHF} \\

	\end{rSection}

\end{document}
